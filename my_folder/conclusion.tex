\chapter*{Заключение} \label{ch-conclusion}
\addcontentsline{toc}{chapter}{Заключение}	% в оглавление 
	В рамках проведённого исследования были выполнены все поставленные задачи. Были изучены алгоритмы, используемые в разработанной архитектуре. Помимо предлагаемой архитектуры был также разработан гибридный алгоритм для вывода полупрозрачных объектов. Был реализован демонстрационный проект, использующий изученные современные алгоритмы компьютерной графики, и поддерживающий динамическую смену архитектуры конвейера вывода.
	
	С использованием демонстрационного проекта был проведён эксперимент, в ходе которого были показаны преимущества предлагаемой архитектуры перед традиционной. Визуальные результаты работы демонстрационного проекта были признаны качественными.
	
	Таким образом, благодаря сохранению основной структуры, предложенная архитектура может быть использована в современных графических системах, тем самым выделяя больше времени центрального процессора на выполнение различных алгоритмов, повышающих уровень интерактивности (например обработка физики твёрдых тел, или поведения искусственного интеллекта).
	
	В качестве дальнейшей работы наиболее интересными представляются следующие два направления:
	
	Первое - отрисовка полупрозрачных объектов. Несмотря на то, что гибридный алгоритм работает достаточно быстро, для его работы требуется создавать списки для каждого пикселя, что может занимать много видеопамяти. В связи с этим, стоит изучить (и возможно разработать) другие алгоритмы для отрисовки полу-прозрачных объектов, не требующие вывода в определённом порядке.

	Второе - дальнейший перенос задач центрального процессора на графический. Благодаря предлагаемому конвейеру, с центрального процессора была снята задача отбрасывания невидимых объектов, однако существуют и другие задачи, которые можно было бы выполнять на графическом процессоре. Например, обработка положений объектов при использовании скелетной анимации.
	
	Несмотря на всё вышеперечисленное, предлагаемая архитектура позволяет выводить сложные, высоко-нагруженные сцены с высоким уровнем реализма и высокой производительностью.