\chapter*{Заключение} \label{ch-conclusion}
\addcontentsline{toc}{chapter}{Заключение}	% в оглавление 
	В рамках проведённого исследования были выполнены все поставленые задачи. Были изучены алгоримы используемые в разработанной архитектуре. Помимо предлагаемой архитектуры был так-же разработан гибридный алгоритм для вывода полу-прозрачных объектов. Был реализован демонстрационный проект, использующий изученные современные алгоритмы компьютерной графики, и поддерживающий динамическую смену архитектуры конвейера вывода.
	
	На демонстрационном проекте был проведён эксперимент, в ходе которого были показаны преимущества предлагаемой архитектуры над традиционной. Визуальные результаты работы демонтрационного проекта были приняты качественными.
	
	Таким образом, благодаря сохранению основной структуры, предложенная архитектура может быть использована в современных графических системах, тем самым выделяя больше времени центрального процессора на выполнение различных алгоритмов, повышающих уровень интерактивности (например просчёт физики твёрдых тел, или поведения искуственного интелекта).
	
	В качестве дальнейшей работы, наиболее интересными представляются следующие два направления: 	
	
	Первое - отрисовка полупрозрачных объектов. Несмотря на то, что гибридный алгоритм работает достаточно быстро, для его работы требуется создавать списки для каждого пикселя, что может занимать много видеопамяти. В связи с этим, стоит изучить/разработать другие алгоритмы для отрисовки полу-прозрачных объектов, не требующих вывода в определённом порядке.

	Второе - дальнейший перенос задач центрального процессора на графический. Благодаря предлагаемому конвейеру, с центрального процессора была снята задача отбрасывания невидимых объектов, однако существуют и другие задачи, которые можно было бы выполнять на графическом процессоре. Например, переподсчёт положений объектов при использовании скелетной анимации.