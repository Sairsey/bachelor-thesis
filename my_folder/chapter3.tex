\chapter{ПРЕДЛАГАЕМОЕ РЕШЕНИЕ} \label{ch3}
Одной из основных целей данной работы было разработать графический конвейер, максимально приближенный к существующим графическим конвейерам. Таким образом будут достигнуты две цели:
\begin{enumerate}[1.] 
	\item Эксперименты, проводимые при помощи разработанной архитектуры, будут содержать те же затратные (с точки зрения производительности) места, которые имеют современные графические системы. А значит, результаты, полученные в ходе экспериментов, будут более приближены к значениям, получаемым в реальных современных графических системах
	\item Будут разработаны и реализованы алгоритмы, которые можно будет использовать другим графическим системам, если они возьмут за основу предлагаемую архитектуру
\end{enumerate}


\section{Непрямая отрисовка} \label{ch3:indirect_draw}
Как было уже упомянуто ранее, основным способом отправки команд на графический процессор считается сбор \say{списка команд} на центральном процессоре и последующая передача получившегося \say{списка} на графический процессор. Однако современных API добавилась возможность дополнить вышеупомянутый \say{список команд} дополнительным массивом команд прямо из графического процессора. Эта технология называется непрямой отрисовкой и в DirectX 12 эту технологию реализует команда ExecuteIndirect. 

Для того чтобы воспользоваться командой ExecuteIndirect, ей необходимо передать следующие параметры:
\begin{enumerate}[1.] 
	\item \say{Сигнатуру вызова} - перечень команд которые необходимо вызвать для каждого рисуемого объекта. Обязательно должна оканчиваться командой отрисовки.
	\item Буфер из видеопамяти с параметрами (которые требует сигнатура), для каждого выводимого объекта.
	\item Количество отрисовываемых объектов (число или указатель на буфер из видеопамяти).
\end{enumerate}

Таким образом, простейший алгоритм отрисовки любого числа объектов будет выглядеть так: (\firef{alg:simpleIndirect})

\begin{algorithm} %[h]
	\SetKwFunction{algoSimpleIndirectPseudocode}{} 
	\SetKwProg{myalg}{Algorithm}{}{} %write in 2nd agrument <<Algorithm>>, <<Procedure>> etc
	\nonl\myalg{\algoSimpleIndirectPseudocode}{
		Загрузить все объекты
		
		Создать сигнатуру
		
		Создать буфер (\textit{SRVbuffer}) и сохранить в него информацию для каждого объекта (например указатель на буфер вершин) 
		
		Установить формат хранения вершин
		
		Установить формат соединения вершин
		
		Установить программу-шейдер, для отрисовки
		
		
		\For {each frame}{
			ExecuteIndirect(\textit{SRVbuffer})
		}	
	}
	\caption{Примерный псевдокод простейшего алгоритма использующего непрямую отрисовку}\label{alg:simpleIndirect}
\end{algorithm}
\FloatBarrier

Таким образом мы добились поставленной цели, и количество команд в \say{списке команд} при данном подходе действительно константно относительно числа объектов. Но на практике этот алгоритм будет работать хуже, чем традиционный. Причина в том, что традиционный подход позволяет заранее отбросить отрисовку части объектов, просто не добавив их в \say{список команд}, а данный (\firef{alg:simpleIndirect}) алгоритм не даёт такой возможности. Однако, это можно было бы исправить, если буфер с параметрами (\textit{SRVbuffer}) мог бы изменяться на каждом кадре работы приложения. Это приводит нас к следующему алгоритму (\firef{alg:IndirectWithCull})

\begin{algorithm} %[h]
	\SetKwFunction{algoIndirectWithCullPseudocode}{} 
	\SetKwProg{myalg}{Algorithm}{}{} %write in 2nd agrument <<Algorithm>>, <<Procedure>> etc
	\nonl\myalg{\algoIndirectWithCullPseudocode}{
		Загрузить все объекты
		
		Создать сигнатуру
		
		Создать буфер (\textit{SRVbuffer}) и сохранить в него информацию для каждого объекта (например указатель на буфер вершин)
		
		Cоздать буфер (\textit{UAVbuffer}), совпадающий размером с предыдущим, 
		
		Cоздать счётчик (\textit{UAVcounter}) 
		
		Установить формат хранения вершин
		
		Установить формат соединения вершин
		
		\For {each frame}{
			
			Установить программу-шейдер, для обработки команд
			
			Обнулить счётчик (\textit{UAVcounter}) 
			
			Запустить вычислительный шейдер, который скопирует из \textit{SRVbuffer} в \textit{UAVbuffer} параметры нужных команд и изменит \textit{UAVcounter} на значение равное числу скопированных команд \label{alg:IndirectWithCull:culling}
			
			Установить программу-шейдер, для отрисовки
			
			ExecuteIndirect(\textit{UAVbuffer}, \textit{UAVcounter})
		}	
	}
	\caption{Примерный псевдокод алгоритма использующего непрямую отрисовку с отбрасыванием команд}\label{alg:IndirectWithCull}
\end{algorithm}
\FloatBarrier

Отметим преимущества и недостатки описанного алгоритма(\firef{alg:IndirectWithCull}):
\begin{enumerate}[1.] 
	\item Преимущество: число вызовов отрисовки в "списке команд" всё ещё является константным, относительно числа объектов
	\item Преимущество: число вызовов отрисовки на графическом процессоре не меняется(см главу \ref{ch2:Programmable-Vertex-Pulling})
	\item Преимущество: используются стандартные буферы, а значит оптимизации программы-драйвера не перестанут работать (см главу \ref{ch2:Programmable-Vertex-Pulling})
	\item Преимущество: нет проблем с фрагментацией (см главу \ref{ch2:Programmable-Vertex-Pulling})
	\item Преимущество: проверки проводимые при определении \say{нужных} команд (см \ref{alg:IndirectWithCull:culling} на \firef{alg:IndirectWithCull}), будут происходить параллельно. А так как обычно графический процессор обладает большим числом вычислительных ядер, чем центральный процессор, предложенный алгоритм будет работать быстрее.
	\item Недостаток: невозможно установить порядок, в котором объекты будут отрисовываться.
\end{enumerate} % Неявная отрисовка

\section{Общая структура} \label{ch3:pipeline_struct}
	Большинство графических приложений имеют схожую структуру конвейера отрисовки, состоящую из 3-х этапов:
	\begin{enumerate}[1.] 
		\item этап Pre-pass - выполнение задач, которые необходимо выполнить до рисования объектов на экран. Например: отбрасывание не видимых объектов, построение карт теней, предварительный подсчёт карты глубины.
		\item этап Render pass - отрисовка всех объектов на кадр.
		\item этап Post process - применение эффектов на получившийся кадр.
	\end{enumerate}
		
	\begin{figure}[ht!] 
		\center
		\includegraphics [scale=0.35] {my_folder/images//unity_pipeline}	
		\caption{Упрощенная схема конвейера отрисовки Unity} 
		\label{fig:unity_pipeline}  
	\end{figure}

	Предлагаемый конвейер сохраняет эту структуру, несмотря на изменения в алгоритме отрисовки, и требует 11 + S (где S - число источников света, генерирующих тень) вызовов отрисовки в \say{списке команд}, что видно по схеме конвейера на \firef{fig:pipeline_schema}. Каждый этап предствляет собой набор из подэтапов, где каждый подэтап решает ровно одну задачу.
	
	\begin{figure}[ht!] 
		\center
		\includegraphics [scale=0.4] {my_folder/images//pipeline_schema}	
		\caption{Схема предлагаемого конвейера. В скобках указано количество вызывов ExecuteIndirect.} 
		\label{fig:pipeline_schema}  
	\end{figure}
	
	\FloatBarrier % Общая структура

\section{Этап Pre-pass} \label{ch3:pre_pass}
	\begin{figure}[ht!] 
		\center
		\includegraphics [scale=0.4] {my_folder/images//prepass_schema}	
		\caption{Схема этапа Pre-pass предлагаемого конвеера.} 
		\label{fig:prepass_schema}
	\end{figure}
	
	Во время отрисовки кадра, на разных подэтапах могут требоваться команды, удовлетворяющие разным критериям. Для этого вводится несколько буферов с параметрами команд для ExecuteIndirect, каждый из которых имеет разное имя, работающее как фильтр:
	\begin{enumerate}[1.]
		\item All - буфер, в котором находятся параметры для всех объектов, причутствующих на сцене.
		\item OpaqueAll - буфер, в котором находятся параметры всех непрозрачных объектов.
		\item TransparentAll - буфер, в котором находятся параметры всех прозрачных объектов.
		\item OpaqueFrustum - буфер, в котором находятся параметры всех непрозрачных объектов, пересекающих трапецию видимости.
		\item OpaqueCulled - буфер, в котором находятся параметры всех непрозрачных объектов, видимых пользователю.
		\item TransparentCulled - буфер, в котором находятся параметры всех прозрачных объектов, видимых пользователю.
	\end{enumerate}
	
	Основной задачей этапа Pre-pass является заполнение всех вышеописанных буферов (кроме All). Разберём подэтапы данного этапа:
	\subsection{Frustum culling} \label{ch3:pre_pass:frustum}
		Данный подэтап отвечает за заполнение буферов OpaqueAll, TransparentAll и OpaqueFrustum при помощи алгоритма с одноименным названием Frustum culling. Основная идея алгоритма заключается в том, чтобы для каждого объекта преверить, пересекает ли он усеченную пирамиду видимости камеры, или нет (см \firef{fig:frustum_culling}). Если объект и пирамида видимости камеры не пересекаются, то объект можно и не выводить.
		\begin{figure}[ht!] 
			\center
			\includegraphics [scale=1] {my_folder/images//frustum_culling}	
			\caption{Схема работы алгоритма Frustum Culling.} 
			\label{fig:frustum_culling}
		\end{figure}
		
		Однако, зачастую, объекты представляют собой сложные невыпуклые геометрические формы, для которых считать пересечения является трудной вычислительной задачей. В связи с этим, в качестве оптимизации, для каждого объекта строится параллельный осям ограничивающий параллелепипед(см. \firef{fig:aabb}), и проверяется пересечение не объектов с усечённой пирамидой видимости, а пересечение параллелепипедов и усечённой пирамиды. 
		
		\begin{figure}[ht!] 
			\center
			\includegraphics [scale=0.8] {my_folder/images//aabb}	
			\caption{Пример построения параллельного осям ограничивающего параллелепипеда в плоском случае.} 
			\label{fig:aabb}
		\end{figure}
		
		Для того чтобы проверить, пересекаются ли параллелипипед и усеченная пирамида, достаточно задать уравнения плоскостей, содержащих грани усеченной пирамиды, и имеющие нормали, направленные \say{внутрь} пирамиды. Предположим, что плоскости задаются уравнениями \ref{eq:frustum_plane}, а вершины параллельного осям ограничивающего параллелепипеда задаются как \ref{eq:aabb_points}.
		
		\begin{equation} % \tag{S} % tag - вписывает свой текст 
			\label{eq:frustum_plane}
			\begin{multlined}
				A_i * x + B_i * y + C_i * z + D = 0, \forallAlt 1 \le i \le 6 
			\end{multlined}
		\end{equation}
		
		\begin{equation} % \tag{S} % tag - вписывает свой текст 
			\label{eq:aabb_points}
			\begin{multlined}
				{x_j, y_j, z_j} \forallAlt 1 \le j \le 8 
			\end{multlined}
		\end{equation}
		
		Тогда выполнение условия \ref{eq:frustum_check} гарантирует то, что объект не попадёт в область видимости.
		
		\begin{equation} % \tag{S} % tag - вписывает свой текст 
			\label{eq:frustum_check}
			\begin{multlined}
				\existsAlt i  \forallAlt j A_i * x_j + B_i * y_j + C_i * z_j + D < 0
			\end{multlined}
		\end{equation}
				
	\subsection{Depth pre-pass} \label{ch3:pre_pass:depth}
		Данный подэтап выполняет задачу построения иерархической карты глубины, используя объекты из буфера OpaqueFrustum.
		
		Для начала, определим что такое карта глубины и что такое иерархическая карта глубины. \textit{Картой глубины} называется монохромное изображение, где для каждого пикселя вместо интенсивности цвета хранится его расстояние до камеры.  \textit{Иерархической картой глубины} называется набор монохромных изображений, удовлетворяющих следующим условиям.
		\begin{enumerate}[1.]
			\item Первое изображение совпадает по размеру с картой глубины.
			\item Ширина и высота каждого следующего изображения меньше ширины и высоты предыдущего в 2 раза.
			\item Первое изображение совпадает с картой глубины.
			\item Для каждого следующего изображения, интенсивнось пикселя считается как максимум из четырёх соотвествующих пикселей предыдущего.
		\end{enumerate}
	
		\begin{figure}[ht!] 
			\center
			\includegraphics [scale=1] {my_folder/images//depth_map}	
			\caption{Пример карты глубины.} 
			\label{fig:depth_map}
		\end{figure}
		
		\begin{figure}[ht!] 
			\center
			\includegraphics [scale=1] {my_folder/images//hier_depth_map}	
			\caption{Пример иерархической карты глубины.} 
			\label{fig:hier_depth_map}
		\end{figure}
		\FloatBarrier
		Построение такой карты открывает следующие преимущества:
		\begin{enumerate}[1.]
			\item Предварительный подсчёт карты глубины позволяет позднее высчитывать цвет пикселей только тех объектов, для которых глубина совпадает с глубиной в карте глубины.
			\item Построенная иерархическая карты глубины позволяет применить алгоритм Occlusion culling, о котором будет сказано далее.
		\end{enumerate}
	\subsection{Occlusion culling} \label{ch3:pre_pass:occlusion}
		Данный подэтап отвечает за заполнение буферов OpaqueCulled и  TransparentCulled при помощи алгоритма Occlusion culling. Основная цель данного алгоритма заключается в отбрасывании тех объектов, которые оказываются перекрыты близлежащими объектами(см \firef{fig:occlusion_culling}). 
		
		\begin{figure}[ht!] 
			\center
			\includegraphics [scale=0.27] {my_folder/images//occlusion_culling}	
			\caption{Схема работы алгорима Occlusion culling.} 
			\label{fig:occlusion_culling}
		\end{figure}
		
		Для этого:
		\begin{enumerate}[1.]
			\item Каждый объект представляется в виде параллельного осям ограничивающего параллелепипеда (см. главу \ref{ch3:pre_pass:frustum})
		 	\item Среди вершин параллелепипеда находится ближайшая, и её расстояние до камеры сохраняется.
		 	\item Каждая вершина параллелепипеда проецируется на экран, тем самым получаются координаты пикселей, соответствующих данной вершине.
		 	\item Вокруг спроецированных вершин строится параллельный осям ограничивающий прямоугольник.
		 	\item По размеру прямоугольника выбирается изображение из иерархической карты глубин(см. главу \ref{ch3:pre_pass:depth}) таким образом, чтобы весь прямоугольник соответствовал одному пикселю изображения.
		 	\item Если интенсивность, в соответствующем пикселе изображения из иерархической карты глубин, меньше, чем сохранённая глубина - то объект перекрывается другими объектами и его копировать не надо. 
		 	\item Иначе, объект может быть виден, и тогда его необходимо скопировать в соответсвующий буфер. 
		\end{enumerate}
		
	\subsection{Построение карт теней} \label{ch3:pre_pass:shadow_maps}
		Данный подэтап отвечает за построение карт теней, для реализации алгоритмов затенения. Сами по себе карты теней очень похожи на \say{карту глубины}(см главу \ref{ch3:pre_pass:depth}). Основное отличие заключается лишь в том, что в интенсивность записывается расстояние не от камеры до объекта, а расстояние от источника света до объекта.
		
		При наличии карты теней, проверить находится ли точка в тени достаточно просто: необходимо лишь посчитать расстояние от точки до источника света, и если оно больше, записанного в карте теней, то в данной точке присутствует тень.
		
		Однако в силу того, что изображение имеет конечное разрешение, при некотором приближении тени могут оказазаться \say{блочными} как показано на \firef{fig:blocky_shadows}.
		 
		 \begin{figure}[ht!] 
		 	\center
		 	\includegraphics [scale=1] {my_folder/images//blocky_shadows}	
		 	\caption{Пример "блочных" теней.} 
		 	\label{fig:blocky_shadows}
		 \end{figure}
		 
		 Чтобы избежать этого эффекта и необходимости затрачивать большое количество памяти на тени, используется алгоритм Persentage Closure Filtering. Суть этого алгоритма заключается в том, чтобы сравнивать расстояние от точки до источника света не с одиним пикселем, а с несколькими, рядом-лежащими пикселями. При применении такого подхода, тени становятся \say{мягких}, как показано на \firef{fig:pcf_shadows}.
		 
		 \begin{figure}[ht!] 
		 	\center
		 	\includegraphics [scale=1] {my_folder/images//pcf_shadows}	
		 	\caption{Пример мягких теней.} 
		 	\label{fig:pcf_shadows}
		 \end{figure}
		 
		 В связи с тем, что невозможно одновременно заполнить карты теней для всех источников света на сцене, этот подэтап требует отрисовки для каждого источника света отбрасывающего тень. % Prepass	

\section{Этап Render pass} \label{ch3:render_pass}
	\begin{figure}[ht!] 
		\center
		\includegraphics [scale=0.4] {my_folder/images//renderpass_schema}	
		\caption{Схема этапа Render pass предлагаемого конвеера.} 
		\label{fig:renderpass_schema}
	\end{figure}
	
	На данном этапе, после всей подготовительной работы, поведённой в предыдущем этапе, происходит отрисовка объектов на кадр.
	
	\subsection{Непрозрачные объекты} \label{ch3:render_pass:opaque}
		Данный подэтап отрисовывает все непрозрачные объекты, используя буфер OpaqueCulled, карту глубины из главы \ref{ch3:pre_pass:depth} и карты теней из главы \ref{ch3:pre_pass:shadow_maps}.
		
		Для отображения объектов с учётом их материала и расположения относительно источников света используются различные модели освещения. В предлагаемой архитектуре используется алгоритм освещения, называющийся Physically based rendering\cite{pharr2016physically}.
	
		\subsubsection{Physically based rendering} \label{ch3:render_pass:opaque:pbr}
			Данный алгоритм освещения использует в своей основе физическую модель микрограней\cite{walter2007microfacet}. В этой физической модели поверхность любого объекта представляет собой множество идеальных зеркал, находящихся под разными углами друг к другу (см \firef{fig:microfacet}).
			
			\begin{figure}[ht!] 
				\center
				\includegraphics [scale=0.4] {my_folder/images//microfacet}	
				\caption{Схема физической модели микрограней. Слева - шершавая поверхность, справа - гладкая} 
				\label{fig:microfacet}
			\end{figure}
			
			Для понимания принципов работы алгоритма Physically based rendering необходимо рассмотреть \say{Основное уравнение рендеринга} (см. формулу \ref{eq:rendering}), предложенное Джеймсом Каджия\cite{kajiya1986rendering}. 
			
			\begin{equation}
				\label{eq:rendering}
				\begin{multlined}
					L_o(x, \omega_o) = L_e(x, \omega_o) + \int_{\Omega} f_r(x, \omega_i, \omega_o)L_i(x, \omega_i)(n_x * \omega_i)d\omega_i
				\end{multlined}
			\end{equation}
			
			Данное уравнение показывает, что интенсивность света в точке $x$ по направлению $\omega_o$ равна сумме излучемой интенсивности ($L_e(x, \omega_o)$) и отраженной интенсивности, где последняя считается как интеграл по полусфере ($\Omega$) произведения интенсивности падающего света ($L_i(x, \omega_i)$), двунаправленной функции отражательной способности ($f_r(x, \omega_i, \omega_o)$) и косинуса угла падения ($(n_x * \omega_i)$). Двунаправленая функция отражательной способности $f_r$ часто называется BRDF функцией. 
			
			\begin{figure}[ht!] 
				\center
				\includegraphics [scale=0.6] {my_folder/images//rendering_eq}	
				\caption{Обозначения используемые в "основном уравнении рендеринга"} 
				\label{fig:base_rendering}
			\end{figure}
			
			В предлагаемом конвейере используется BRDF функция Кука-Торренса\cite{cook1982reflectance}.
		%TODO: \subsubsection{Image based lighting} \label{ch3:render_pass:opaque:ibl}
	\subsection{Skybox} \label{ch3:render_pass:skybox}
		На данном подэтапе в кадр выводится фоновое изображение, на котором изображено окружение сцены. Данный подэтап рисуется после отрисовки непрозрачных объектов, чтобы уменьшить число перерисовываемых пикселей кадра, так как фоновое изображение рисуется только в тех пикселях, где не были отрисованы объекты. 
		
		Описанное фоновое изображение представляет собой кубическую карту. \textit{Кубической картой} называется набор из 6 изображений, снятых с 6-ти направлений и расположенных в развёртке куба, как показано на \firef{fig:skybox}.
		
		\begin{figure}[ht!] 
			\center
			\includegraphics [scale=0.8] {my_folder/images//skybox}	
			\caption{Пример кубической карты окружения} 
			\label{fig:skybox}
		\end{figure}
		
		Таким образом, при необходимости вывести цвет в пикселе, в котором не были отрисованы объекты, достаточно будет посчитать направление, в котором этот пиксель находится и взять точку, соответствующую точке на кубе, в указаном направлении (см. \firef{fig:cube_sample}).
		
		\begin{figure}[ht!] 
			\center
			\includegraphics [scale=1] {my_folder/images//cube_sample}	
			\caption{Схема работы вычисления цвета пикселя c использованием кубической карты} 
			\label{fig:cube_sample}
		\end{figure}
		
	\subsection{Полу-прозрачные объекты} \label{ch3:render_pass:transparents}
		На данном подэтапе в кадр выводятся полупрозрачные объекты при помощи буфера TransparentCulled. Как понятно из названия, полу-прозрачные объекты отличаются от непрозрачных тем, что через них можно видеть объекты находящиеся позади. Из-за этого нельзя воспользоваться картой глубины (cм. главу \ref{ch3:pre_pass:depth}) для отрисовки только ближайшего пикселя, что может привести к ситуации, приведённой на \firef{fig:incorrect_transparent}.
		
		\begin{figure}[ht!] 
			\center
			\includegraphics [scale=0.5] {my_folder/images//incorrect_transparent}	
			\caption{Пример отрисовки прозрачных объектов в неправильном порядке} 
			\label{fig:incorrect_transparent}
		\end{figure}
		\FloatBarrier
		
		Чтобы избежать изображенной ситуации, необходимо производить сортировку объектов. Тогда, если выводить объекты начиная с самого дальнего, то цвет в пикселе можно вычислять по формуле \ref{eq:blend-formula}.
		
		\begin{equation}
			\label{eq:blend-formula}
			\begin{multlined}
				C_{new} = \alpha * C_{transparent} + (1 - \alpha) * C_{pixel}
			\end{multlined}
		\end{equation}
		
		Где:
		\begin{enumerate}[1.]
			\item $C_{new}$ - новый цвет пикселя
			\item $C_{transparent}$ - цвет полученный в результате применения алгоритма освещения, для данного пикселя
			\item $C_{pixel}$ - цвет, хранящийся в данном пикселе.
			\item $\alpha$ - коэффициент прозрачности. Значение 0 соответствует абсолютно прозрачному объекту, а значение 1 соответствует абсолютно непрозрачному.
		\end{enumerate}
		
		Однако предлагаемый алгоритм неявной отрисовки (см. главу \ref{ch3:indirect_draw}) не позволяет установить порядок отрисовки объектов. Из-за этого необходимо использовать особые алогритмы отрисовки прозрачных объектов.		
		
		\subsubsection{Order Independent Transparency} \label{ch3:render_pass:transparents:oit}
			Первым из рассмотренных алгоритмов является алгоритм Order Independent Transparency with per-pixel linked lists \cite{barta2011order}. В данном алгоритме для каждого пикселя экрана заводится список, в каждом элементе которого хранится: цвет, глубина и коэффициент прозрачности. Далее алгоритм работает в 2 запуска отрисовки:
			
			\begin{enumerate}[1.]
				\item Отрисовываются все полупрозрачные объекты, но результат отрисовки объекта записывается не в кадр, а добавляется в конец созданных списков
				\item На экран отрисовывается прямоугольник, покрывающий весь экран. Для каждого пикселя прямоугольника берётся соответсвующий ему список и значения в этом списке сортируются по глубине. Далее, используя отсортированный список, последовательно применим формулу \ref{eq:blend-formula} и получим формулу для итогового цвета \ref{eq:oit-formula}
			\end{enumerate}
			
			\begin{figure}[ht!] 
				\center
				\includegraphics [scale=0.5] {my_folder/images//first_step_oit}	
				\caption{Изображение демонстрирующее первый этап работы алгоритма} 
				\label{fig:first_step_oit}
			\end{figure}
			 
			\begin{equation}
				\label{eq:oit-formula}
				\begin{multlined}	 
			 		C_{out} = C_{1}\alpha_1 + \sum_{i=2}^{N}(C_i\alpha_i\prod _{j=1}^{i - 1}(1 - \alpha_j)) + 
			 		C_{opaque}\prod _{j=1}^{N}(1 - \alpha_j)   
			 	\end{multlined}
			 \end{equation}
			
			Где:
			\begin{enumerate}[1.]
				\item $N$ - размер списка для данного пикселя
				\item $C_{out}$ - результирующий цвет пикселя
				\item $C_{opaque}$ - цвет непрозрачного объекта, полученный из пикселя кадра.
				\item $C_{i}$ - цвет, полученный из элемента отсортированного списка с номером $i$.
				\item $\alpha_i$ - коэффициент прозрачности, полученный из элемента отсортированного списка с номером $i$.
			\end{enumerate}			
			
			Нетрудно заметить, что благодаря переносу сортировки на второй этап, появляется возможность отрисовывать объекты на первом этапе в любом порядке. Однако время, требуемое на выполнение вышеописанной сортировки, сильно сказывается на производительности. Авторами статьи упоминается, что количество отрисовываемых кадров в секунду при использовании данного подхода, падает с 110 вплоть до 5. Это означает, что запуск отрисовки с сортировкой может занимать до 190 миллисекунд, что неприемлемо для современных приложений реального времени.
		\subsubsection{Weighted Blended Order Independent Transparency} \label{ch3:render_pass:transparents:wboit}
			В 2013 году, в качестве улучшения предыдущего алгоритма, был представлен алгоритм Weighted Blended Order Independent Transparency\cite{mcguire2013weighted}. Данный алгоритм повторяет идею предыдущего алгоритма, однако, вместо сортировки и применения формулы \ref{eq:oit-formula}, предлагается использовать её аппроксимацию \ref{eq:wboit-formula}
					 
			\begin{equation}
				\label{eq:wboit-formula}
				\begin{multlined}	 
					C_{out} = \frac{\sum_{i=1}^{N}C_i}{\sum_{i=1}^{N}\alpha_i}(1 - \prod_{i=1}^{N}\alpha_i) + 
					C_{opaque}\prod _{j=1}^{N}(1 - \alpha_j)   
				\end{multlined}
			\end{equation}
			
			Очевидно, что данная аппроксимация работает гораздо быстрее подхода, описаного в оригинальном алгоритме. Однако, при коэффициентах $\alpha_i$ близких к единице, погрешности аппроксимации становятся явно заметны человеческому глазу. Визуальное сравнение можно увидеть на \firef{fig:oit_vs_wboit}.
			
			\begin{figure}[!htbp]
				\centering
				\begin{subfigure}[b]{0.3\textwidth}
					\centering
					\includegraphics[width=\textwidth]{my_folder/images//oit_flower}
					\caption{Изображение полученное алгоритмом OIT.\linebreak Время построения кадра: 6.2ms}
					\label{fig:oit_flower}
				\end{subfigure}
				\begin{subfigure}[b]{0.3\textwidth}
					\centering
					\includegraphics[width=\textwidth]{my_folder/images//wboit_flower}
					\caption{Изображение полученное алгоритмом WBOIT.\linebreak Время построения кадра: 1.9ms}
					\label{fig:wboit_flower}
				\end{subfigure}				
				\captionsetup{justification=centering} %центрировать
				\caption{Сравнение работы алгоритмов OIT и WBOIT}\label{fig:oit_vs_wboit} 
			\end{figure}
			
		\subsubsection{Hybrid Order Independent Transparency} \label{ch3:render_pass:transparents:hybrid_oit}
			В результате изучения описаных алгоритмов, было решено разработать новый, гибридный алгоритм, совмещающий в себе преимущества описанных алгоритмов. Как и предыдущие алгоритмы, он состоит из двух запусков отрисовки:
			
			\begin{enumerate}[1.]
				\item Как и в предыдущих алгоритмах, все полупрозрачные объекты отрисовываются в списки, созданные для каждого пикселя.
				\item В списках для каждого пикселя, вместо сортировки всего списка, находятся $K$ наименьших по параметру глубины, и они сортируются между собой. Затем результирующий цвет вычисляется по формуле \ref{eq:hybrid-formula}
			\end{enumerate}
			
			\begin{equation}
				\label{eq:hybrid-formula}
				\begin{multlined}	 
					C_{out} = C_{1}\alpha_1 +
					\sum_{i=2}^{K}(C_i\alpha_i\prod _{j=1}^{i - 1}(1 - \alpha_j)) + \\
					\frac{\sum_{i=K+1}^{N}C_i}{\sum_{i=K+1}^{N}\alpha_i}(1 - \prod_{i=K+1}^{N}\alpha_i)\prod _{i=1}^{K}(1 - \alpha_i) + \\	
					C_{opaque}\prod _{j=1}^{N}(1 - \alpha_j)   
				\end{multlined}
			\end{equation}
			
			Как можно заметить, данный алгоритм смешивает $N-K$ элементов списка по формуле \ref{eq:wboit-formula}, а затем, считая получившийся цвет, как элемент списка с номером $K+1$, применяет формулу \ref{eq:oit-formula}, считая что список состоит из $K+1$ элемента. Визуальное сравнение всех трех методов можно увидеть на \firef{fig:oit_vs_wboit_vs_hybrid}.
		
			\begin{figure}[!htbp]
				\centering
				\begin{subfigure}[b]{0.3\textwidth}
					\centering
					\includegraphics[width=\textwidth]{my_folder/images//oit_flower}
					\caption{Изображение полученное алгоритмом OIT.\linebreak Время построения кадра: 6.2ms}
					\label{fig:oit_flower}
				\end{subfigure}
				\begin{subfigure}[b]{0.3\textwidth}
					\centering
					\includegraphics[width=\textwidth]{my_folder/images//hybrid_flower}
					\caption{Изображение полученное алгоритмом Hybrid OIT.\linebreak Время построения кадра: 2.1ms}
					\label{fig:hybrid_flower}
				\end{subfigure}			
				\begin{subfigure}[b]{0.3\textwidth}
					\centering
					\includegraphics[width=\textwidth]{my_folder/images//wboit_flower}
					\caption{Изображение полученное алгоритмом WBOIT.\linebreak Время построения кадра: 1.9ms}
					\label{fig:wboit_flower}
				\end{subfigure}
			\captionsetup{justification=centering} %центрировать
			\caption{Сравнение работы алгоритмов OIT, WBOIT и Hybrid OIT}\label{fig:oit_vs_wboit_vs_hybrid} 
		\end{figure} % Render pass

\section{Этап Post-process} \label{ch3:post_process}
	\subsection{Tonemapping, HDR и LDR} \label{ch3:post_process:hdr_ldr_tonemapping} % Render pass

%% Вспомогательные команды - Additional commands
%
%\newpage % принудительное начало с новой страницы, использовать только в конце раздела
%\clearpage % осуществляется пакетом <<placeins>> в пределах секций
%\newpage\leavevmode\thispagestyle{empty}\newpage % 100 % начало новой страницы